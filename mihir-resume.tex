%%%%%%%%%%%%%%%%%%%%%%%%%%%%%%%%%%%%%%%
% Deedy CV/Resume
% XeLaTeX Template
% Version 1.0 (5/5/2014)
%
% This template has been downloaded from:
% http://www.LaTeXTemplates.com
%
% Original author:
% Debarghya Das (http://www.debarghyadas.com)
% With extensive modifications by:
% Vel (vel@latextemplates.com)
%
% License:
% CC BY-NC-SA 3.0 (http://creativecommons.org/licenses/by-nc-sa/3.0/)
%
% Important notes:
% This template needs to be compiled with XeLaTeX.
%
%%%%%%%%%%%%%%%%%%%%%%%%%%%%%%%%%%%%%%

\documentclass[letterpaper]{deedy-resume} % Use US Letter paper, change to a4paper for A4 
\usepackage{epstopdf}
\begin{document}

%----------------------------------------------------------------------------------------
%	TITLE SECTION
%----------------------------------------------------------------------------------------

%\lastupdated % Print the Last Updated text at the top right

\namesection{Mihir}{Shete}{ % Your name
%\urlstyle{same}\url{http://linkedin.com/in/mihirshete} \\ % Your website, LinkedIn profile or other web address

\href{mailto:mihirsht@gmail.com}{mihirsht@gmail.com} | +91-7893217241 \\% Your contact information
}

%----------------------------------------------------------------------------------------
%	LEFT COLUMN
%----------------------------------------------------------------------------------------

\begin{minipage}[t]{0.33\textwidth} % The left column takes up 33% of the text width of the page

%------------------------------------------------
% Links
%------------------------------------------------

\section{Connect @}

\includegraphics[height=10pt]{Linkedin-256.png} LinkedIn:  \href{https://www.linkedin.com/in/mihirshete}{mihirshete} \\
\includegraphics[height=10pt]{GitHub-Mark-32px.png} Github: \href{https://github.com/smihir}{smihir} \\
\includegraphics[height=10pt]{Facebook-256.png} Facebook:  \href{https://www.facebook.com/mihirshete}{mihirshete} \\

\sectionspace % Some whitespace after the section

%------------------------------------------------
% Skills
%------------------------------------------------

\section{Skills}

\subsection{Programming}
C \textbullet{} Lua \textbullet{} Shell \textbullet{} Matlab \\
Octave \textbullet{} Python \textbullet{} Rails \\
Assembly \textbullet{} \LaTeX
\vspace{\topsep}
\subsection{Others}
802.11 \textbullet{} Networking \textbullet{} \\
Linux Device Drivers

\sectionspace % Some whitespace after the section

%------------------------------------------------
% Opensource project links
%------------------------------------------------

\section{Project Links}
\subsection{Qualcomm}
\footnotesize \textit{\textbf{(Opensource contributions only)} } \\
\href{https://www.codeaurora.org/cgit/quic/la/platform/vendor/qcom-opensource/wlan/prima/}{Prima Wlan Driver} \\
\href{https://www.codeaurora.org/cgit/quic/la/kernel/msm-3.10}{Qualcomm MSM Kernel} \\
\subsection{Cypress Semiconductor}
\href{https://github.com/smihir/PSoC-BP-Monitor}{Blood Pressure Monitor} \\

\sectionspace % Some whitespace after the section

%------------------------------------------------
% Coursework
%------------------------------------------------

\section{Coursework}

\subsection{Undergraduate}

\textbullet{} Embedded System Design \\
\textbullet{} Medical Instrumentation \\
\textbullet{} Digital Electronics and \\ \hphantom{\textbullet{}}Computer Organization \\
\textbullet{} Microprocessor Programming and \\ \hphantom{\textbullet{}}Interfacing \\
\textbullet{} Analog Electronics \\
\hphantom{\textbullet{}}{\footnotesize \textit{\textbf{(Professional Assistant) }}} \\
\textbullet{} Computer Programming I/II \\
\textbullet{} Probability and Statistics \\
\textbullet{} Control Systems \\

\subsection{Qualcomm}
\textbullet{} ARMv8 Architecture and Design \\
\sectionspace % Some whitespace after the section


%----------------------------------------------------------------------------------------

\end{minipage} % The end of the left column
\hfill
%
%----------------------------------------------------------------------------------------
%	RIGHT COLUMN
%----------------------------------------------------------------------------------------
%
\begin{minipage}[t]{0.66\textwidth} % The right column takes up 66% of the text width of the page

%------------------------------------------------
% Experience
%------------------------------------------------

\section{Experience}

\runsubsection{Qualcomm}
\descript{| Engineer}

\location{July 2014 – Till Date | Hyderabad, India}
\vspace{\topsep} % Hacky fix for awkward extra vertical space
\begin{tightitemize}
\item A member of Wireless Connectivity group, we are involved in design and development of firmware and Linux drivers for Qualcomm's wireless chipsets.
\item I primarily work on  Qualcomm's opensource \textbf{Prima} driver and I am currently the maintainer of data path.
\item Worked with the Linux community to design and develop the regulatory framework in Prima driver.
\item Developed good understanding of new and upcoming wireless protocols and standards like WFA's - WPS, P2P, WMM and IEEE 802.11ac/p
\end{tightitemize}

\sectionspace % Some whitespace after the section

%------------------------------------------------

\runsubsection{TeamF1 Networks}
\descript{| Software Engineering Intern}

\location{May 2010 – July 2011 | Hyderabad, India}
\begin{tightitemize}
\item Design and Development of Linux Device drivers for 802.11 Wireless SoCs in Enterprise Routers and Access Points.
\item Developed an excellent understanding of 802.11/a/b/g/n/i/r.
\item Worked on securing the WiFi Alliance's 802.11n certification for interoperability.
\item Involved in design and development of 802.11r solution independent of the available open source alternatives.
\item Designed and Developed the RSTP functionality for Access Points.
\end{tightitemize}

\sectionspace % Some whitespace after the section

%------------------------------------------------

\runsubsection{Cypress Semiconductor}
\descript{| Trainee}

\location{July 2009 – December 2009 | Chennai, India}
\begin{tightitemize}
\item Design and development of Blood Pressure Monitor.
\item PCB design using EDA tools like Capture CIS and Allegro
\item Gained good insight into working of SoCs and process that go in design and development of a good embedded system.
\end{tightitemize}

\sectionspace % Some whitespace after the section

%----------------------------------------------------------------------------------------

\runsubsection{Mahindra \& Mahindra}
\descript{| Summer Intern}

\location{June 2008 – July 2008 | Nagpur, India}
\begin{tightitemize}
\item Did a detailed study of the Tractor Assembly Line.
\item Drafted the Electrical Line Diagram of the Tractor Assembly Plant.
\end{tightitemize}

\sectionspace % Some whitespace after the section

%------------------------------------------------
% Education
%------------------------------------------------

\section{Education}

\subsection{BITS - Pilani, Goa Campus}

\descript{B.E(Hons.) Electronics and Instrumentation}
\location{Ausgust 2006 – May 2010 | Goa, India}
\location{Cum. GPA 8.00/10}
\begin{tightitemize}
\item Concentrated on projects and internships in the areas of Embedded systems' design and development.
\item As a member of PSoC club which was newly founded, was involved in designing training material which is used by Cypress Semiconductor as the official Lab Book for the University Alliance Program.
\end{tightitemize}

\sectionspace % Some whitespace after the section

\subsection{Shri Shivaji Science College}
\location{Grad. May 2006 | Nagpur, India}
\location{Score 92.53\%}

\sectionspace % Some whitespace after the section

\end{minipage} % The end of the right column

%----------------------------------------------------------------------------------------
%	SECOND PAGE
%----------------------------------------------------------------------------------------

\newpage % Start a new page

\begin{minipage}[t]{0.33\textwidth} % The left column takes up 33% of the text width of the page

\section{Links}
\footnotesize \textit{\textbf{(Academic Project Reports)} } \\
\textbullet{} \href{https://github.com/smihir/BP-Monitoring-Algorithms}{Design and Development of \\ \hphantom{\textbullet{}}Blood Pressure Monitor} \\
\textbullet{} \href{https://github.com/smihir/Safety-Critical-Systems}{Safety Critical Systems and Design} \\

\sectionspace % Some whitespace after the section

\end{minipage} % The end of the left column
\hfill
\begin{minipage}[t]{0.66\textwidth} % The right column takes up 66% of the text width of the page

%------------------------------------------------
% Research
%------------------------------------------------

\section{Academic Projects \& Research}

\runsubsection{Design and Development of Blood Pressure Monitor}
\location{Jan 2010 – April 2010 | Goa, India}

\vspace{\topsep} % aaand here we go again...
\begin{tightitemize}
\item Worked with \textbf{\href{http://in.linkedin.com/in/aniketsachan}{Aniket Sachan}} and \textbf{\href{http://in.linkedin.com/pub/vikash-sharma/16/865/aba}{Vikash Sharma}} under the guidance of \textbf{Sarita Kumari} to continue the work done by me in Cypress Semiconductor.
\item We concentrated more on developing simulation of circuits and filters involved in the designing of a BP Monitor so that we can improve upon them.
\item Another aspect of this project was to do a detailed analysis of the available BP calculating algorithms and implement them in Matlab/Octave so that they can be studied in detail and be improved for better accuracy.
\end{tightitemize}

\sectionspace % Some whitespace after the section

%------------------------------------------------

\runsubsection{Design and Verification of Safety Critical Systems} \\
\location{Jan 2010 – April 2010 | Goa, India}
\begin{tightitemize}
\item The 6 months in Cypress' Medical engineering group fostered the importance of considering safety while designing systems that are used in critical areas and I took this study project under the guidance of \textbf{\href{http://universe.bits-pilani.ac.in/goa/anupkr/profile}{Dr. K.R Anupama}}.
\item This project was to study the established standards for designing safety critical systems and provide guidelines and introduction to formal methods for building Safety Critical Systems.
\end{tightitemize}

\sectionspace % Some whitespace after the section

%------------------------------------------------
% Awards
%------------------------------------------------

\section{Awards \& Recognition}

\begin{tabular}{rll}
2010 & 2\textsuperscript{nd} & Symphony Event held in Quark '10(techfest of BITS, Goa)\\
2006 & National & Placed 23\textsuperscript{rd} in the Merit List prepared by the Maharashtra State Board for Secondary and Higher Secondary Education for Higher Secondary School Examinationi held in March 2006\\
2006 & National & Placed 921\textsuperscript{st} in All India Engineering Entrance Examinations \\
\end{tabular}

\sectionspace % Some whitespace after the section

%----------------------------------------------------------------------------------------

\end{minipage} % The end of the right column


\end{document}
